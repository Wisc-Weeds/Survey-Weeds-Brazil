% Options for packages loaded elsewhere
\PassOptionsToPackage{unicode}{hyperref}
\PassOptionsToPackage{hyphens}{url}
%
\documentclass[
  12pt,
  a4paper]{article}
\usepackage{lmodern}
\usepackage{amssymb,amsmath}
\usepackage{ifxetex,ifluatex}
\ifnum 0\ifxetex 1\fi\ifluatex 1\fi=0 % if pdftex
  \usepackage[T1]{fontenc}
  \usepackage[utf8]{inputenc}
  \usepackage{textcomp} % provide euro and other symbols
\else % if luatex or xetex
  \usepackage{unicode-math}
  \defaultfontfeatures{Scale=MatchLowercase}
  \defaultfontfeatures[\rmfamily]{Ligatures=TeX,Scale=1}
\fi
% Use upquote if available, for straight quotes in verbatim environments
\IfFileExists{upquote.sty}{\usepackage{upquote}}{}
\IfFileExists{microtype.sty}{% use microtype if available
  \usepackage[]{microtype}
  \UseMicrotypeSet[protrusion]{basicmath} % disable protrusion for tt fonts
}{}
\makeatletter
\@ifundefined{KOMAClassName}{% if non-KOMA class
  \IfFileExists{parskip.sty}{%
    \usepackage{parskip}
  }{% else
    \setlength{\parindent}{0pt}
    \setlength{\parskip}{6pt plus 2pt minus 1pt}}
}{% if KOMA class
  \KOMAoptions{parskip=half}}
\makeatother
\usepackage{xcolor}
\IfFileExists{xurl.sty}{\usepackage{xurl}}{} % add URL line breaks if available
\IfFileExists{bookmark.sty}{\usepackage{bookmark}}{\usepackage{hyperref}}
\hypersetup{
  pdftitle={Assessment of Weed Management Strategies Prior to Introduction of Auxin-Tolerant Crops in Brazil},
  hidelinks,
  pdfcreator={LaTeX via pandoc}}
\urlstyle{same} % disable monospaced font for URLs
\usepackage[margin=1in]{geometry}
\setlength{\emergencystretch}{3em} % prevent overfull lines
\providecommand{\tightlist}{%
  \setlength{\itemsep}{0pt}\setlength{\parskip}{0pt}}
\setcounter{secnumdepth}{-\maxdimen} % remove section numbering
% Allowing for landscape pages
\usepackage{lscape}
\usepackage{booktabs}
\usepackage{multirow}
\usepackage{float}
\usepackage{placeins}
\newcommand{\blandscape}{\begin{landscape}}
\newcommand{\elandscape}{\end{landscape}}
\usepackage{lineno}
\usepackage{setspace}
\usepackage{ragged2e}




% Left justification of the text: see https://www.sharelatex.com/learn/Text_alignment
% \usepackage[document]{ragged2e} % already in the latex template
\newcommand{\bleft}{\begin{flushleft}}
\newcommand{\eleft}{\end{flushleft}}

\title{Assessment of Weed Management Strategies Prior to Introduction of
Auxin-Tolerant Crops in Brazil}
\author{true \and true \and true \and true}
\date{}

\begin{document}
\maketitle

\textbf{Short title}: Survey of Weed Management in Brazil

\singlespace

\vspace{2mm}\hrule

\textbf{Abstract}: A stakeholder survey was conducted to document
current agricultural practices and perceptions of crop and weed
management challenges across Brazil. The dominant crops managed by
survey respondents are soybean (73\%) and corn (66\%). Approximately
75\% of survey respondents grow or manage annual cropping systems with
two to three crops cultivated per year in succession. Eighteen percent
of respondents manage only irrigated cropping-systems, and over 60\% of
respondents use no-till as a standard practice. According to
respondents, the top five troublesome weed species in Brazilian cropping
systems are \emph{Conyza} spp., \emph{Digitaria insularis},
\emph{Ipomoea} spp., \emph{Eleusine indica}, and \emph{Commelina} spp.
Amongst the eight species documented to have evolved resistance to
EPSPS-inhibitor (glyphosate) in Brazil, \emph{Conyza} spp. and \emph{D.
insularis} were reported as the most concerning weeds. Other than
glyphosate, 31 and 78\% of respondents manage ACCase and/or
ALS-inhibitors resistant weeds, respectively. Besides herbicides, 45\%
of respondents use mechanical, and 75\% use cultural weed control
strategies. Sixty-one percent of survey respondents adopt cover crops to
some extent to suppress weeds and improve soil chemical and physical
properties. Nearly 60\% of survey respondents intend to adopt the
dicamba or 2,4-D resistant crops when available in the country.
According to 54\% of survey respondents, industry representatives are
the main source for crop and weed management information and
recommendations. Herein we present an overview of crop and weed
management practices adopted and challenges faced in Brazilian
agriculture. Results may help practitioners, academics, industry and
policy makers better understand the bad and the good of current cropping
systems and weed management practices adopted in Brazil, and adjust
research, education, technologies priorities and needs moving forward.

\vspace{3mm}\hrule

\textbf{Keywords}: Cover crops; Dicamba; Herbicide weed resistance;
No-till; Soybean; Survey.

\textbf{Nomemclature}: 2,4-D, 2,4-dichlorophenoxyacetic acid;
5-enolpyruvylshikimate-3-phosphate synthase (EPSPS) inhibitor;
Acetolactate synthase (ALS)-inhibitorinhibitor; Acetyl CoA carboxylase
(ACCase) Inhibitor; dicamba; marestail, \emph{Conyza} spp., sourgrass,
\emph{Digitaria insularis}, morningglory, \emph{Ipomoea} spp.;
goosegrass, \emph{Eleusine indica}, Asiatic dayflower, \emph{Commelina}
spp; common beans, \emph{Phaseolus vulgaris}; corn, \emph{Zea mays} L.;
cotton, \emph{Gossypium hirsutum}; soybean, \emph{Glycine max} L. Merr.;
wheat, \emph{Triticum aestivum}.

\doublespace

\bleft
\newpage

Synthetic inputs are used to minimize the impact of pests on crop yield
potential.

\hypertarget{introduction}{%
\section{Introduction}\label{introduction}}

Agriculture has undergone major evolution in the past century leading to
a significant increase in crop yields (Warren 1998). From 1930s to
2020s, grain corn, cotton, rice and soybeans have experienced a crop
yield increase of 740\%, 390\%, 350\% and 290\%, respectively (USDA-NASS
2019, Warren 1998). The discovery of synthetic herbicides, including
MCPA (2-methyl-4-chlorophenoxyacetic acid) and 2,4-D
(2,4-dichlorophenoxyacetic acid) in the 1940s had a positive impact on
crop yields by reducing weed infestations in cropping-systems (Troyer
2001). For example, 2,4-D was adopted as an effective (\textgreater90\%)
broadleaf weed control compound used at lower concentrations compared to
organic herbicides, such as sodium chlorate and sodium thiocyanate
(Marth and Mitchell 1944). The introduction of \emph{S}-triazine (e.g.,
atrazine) represents another milestone in terms of weed control and
herbicide popularity amongst growers (McFarland and Burnside 2011). The
combination of preemergence (PRE) and postemergence (POST) herbicides
plus cultural and mechanical methods reduced the need for
labor-intensive hand weeding, increased efficacy and greatly reduced the
costs for weed management (Gianessi and Reigner 2007, J. T. Holstun et
al. 1960)

From the 1940s to the 1980s (herbicide discovery era), novel herbicide
chemistries with broad weed control spectra, application window in
relation to crop developmental stage, and selectivity were discovered.
During that time, a herbicide sites of action (SOA) were introduced
every three years on average (Appleby 2005). Herbicides quickly became
the synonymous of weed management and through this date represent the
most commonly adopted tool for weed control in conventional production
systems. Given the shortage of and challenges related to novel herbicide
discovery (Duke 2012), industry focus has shifted towards biotechnology
and the production of crop hybrids or varieties genetically engineered
with herbicide-resistant (HR) genes (Bonny 2011, Owen 2000). In 1996,
glyphosate-resistant (GR) soybean (Roundup ready) was the first HR crop
to be introduced, which allowed growers to spray glyphosate, a systemic,
non-selective and very effective herbicide POST in GR-soybean crops
(Padgette et al. 1995--1AD). The success of GR-soybean led to the
introduction of other GR crops (e.g., cotton and corn). Glyphosate use
has risen 15-fold worldwide since the introduction of GR crops in 1996
(Benbrook 2016). This increase was accelerated with introduction of GR
crops to developing countries such as Brazil and Argentina in the early
2000s. In 2014, glyphosate represented 66\% of herbicide applications in
Brazil (SIDRA-IBGE 2020). The GR crops were documented as the most
adopted technology of modern agriculture (Green 2018). However,
glyphosate overreliance resulted in weed shifts and evolution of
GR-weeds (Owen 2008). Thus far, there are 48 GR-resistance weed species
worldwide, being 9 GR-resistance weed species in Brazil, including
\emph{Digitaria insularis} and \emph{Conyza} spp. (Heap 2020).

Rapid evolution of GR-resistance weeds prompted the development of other
HR crops such as glufosinate, 2,4-D or dicamba-resistant (DR) soybean,
cotton and corn. The new synthetic auxin-resistance (AR) technology was
introduced in 2017 to the United States and it will be soon available to
Brazil. The 2,4-D technology is marketed as Enlist E3 (Corteva
Agriscience, Wilmington, DE), which allows glyphosate, glufosinate, and
a new 2,4-D-choline salt formulation application on Enlist crops (Wright
et al. 2010). Moreover, the Roundup Ready 2 Xtend (Bayer Crop Science,
St.~Louis, MO) allows the use of glyphosate and new dicamba
formulations, including diglycolamine salt with VaporGrip, an acetic
acid-acetate buffering system, or a dicamba salt N,N-Bis-(3-aminopropyl)
methylamine) on DR-crops. These new 2,4-D and dicamba formulations are
products with reduced volatility compared to their previous
formulations. However, in the first year of DR crops in the United
States, it was estimated near 1.4 million ha with dicamba injury on
non-DR soybeans (Hager 2018). In Nebraska, 51\% of survey respondents
noted dicamba injury in their non-DR soybeans in 2017 (Werle et al.
2018). It is still controversial whether the injury on sensitive
vegetation is due to dicamba vapor, particle drift and/or tank
contamination. Nonetheless, the upcoming introduction of AR-crops in
Brazil raises concerns of off-target movement (OTM), and it requires
further investigations.

The introduction of synthetic auxin-resistant (AR) crops increases
complexity but represents a new milestone in terms of weed management;
thus documenting current practices prior their introduction and after
nearly 20 years of GR soybean use in Brazil is necessary. Surveys are
useful tools for documenting agricultural practitioners' knowledge and
perceptions regarding specific strategies. For example, a survey with
pesticide applicators indicated the need for further education regarding
application of synthetic auxin technologies in Missouri, USA (Bish and
Bradley 2017). Also, a survey showed that weed control is based on
empiric short term decisions with \textgreater{} 53\% using solely
herbicides for weed management in Argentina (Scursoni et al. 2019). A
survey documented a concern on protoporphyrinogen oxidase
(PPO)--resistant \emph{Amaranthus palmeri} in Southwestern United States
, and the need to diversify weed management focusing on cover crop
research in that geography (Schwartz-Lazaro et al. 2018). Documenting
current weed management practices in different regions of Brazil could
improve weed management decisions, policy, education, investments,
research priorities, and further needs.

In Brazil, growers rely mainly on crop advisors for crop management
decisions, including strategies for weed control (dos Reis MR, personal
communication). The use of survey questionnaires in Brazil with
agricultural practitioners has been lacking. Therefore, the objective of
this survey was to understand from growers and crop advisors (e.g., crop
consultants, coop, industry, and University representatives) current
agricultural management practices, perceptions, and challenges regarding
current cropping-systems and weed management in Brazil. The survey had a
specific focus on troublesome and HR weeds, and evaluating interest,
value and potential challenges the new auxin resistant technologies will
face if deployed/adopted in Brazil.

\hypertarget{material-and-methods}{%
\section{Material and Methods}\label{material-and-methods}}

A survey was developed to understand Brazilian stakeholders' perceptions
and challenges about cropping-systems and weed management strategies
(Supp. file). To reach a uniform representation, the survey was
conducted online using Qualtrics linked to the University of
Wisconsin-Madison and circulated via social media, including
Twitter\textsuperscript \textregistered,
Facebook\textsuperscript \textregistered,
LinkedIn\textsuperscript \textregistered, and
Whatsapp\textsuperscript \textregistered. The messenger
Whatsapp\textsuperscript \textregistered is popular amongst agricultural
stakeholders in Brazil. Moreover, Extension agents assisted with
distributing the survey questionnaire to stakeholders.

The survey comprised three sections. Questions in the first section
focused on respondents' demographics: a) region, b) managed area (ha)
and c) role (e.g., grower or industry rep). The second section was
designed to focus on cropping-systems practices: a) managed crops, b)
crop succession, c) tillage, d) irrigation, e) cover crops and f)
crop-livestock integration. The third section focused on weed management
strategies: a) herbicide program, b) troublesome weeds, c) herbicide
resistant weeds, d) integrated weed management, e) adoption of AR crops
and f) herbicide application decisions. The third section also
incorporated general questions about cropping systems and weed
management challenges.

The online survey was available from April 1 through June 30, 2018.
Results were exported from Qualtrics as a Microsoft Excel (Microsoft
Office, Redmond, WA) file with the answers to each question in separate
columns. Survey data were sorted and analyzed using the \emph{sort},
\emph{filter}, and \emph{count} functions in Microsoft Excel and
\emph{summarise}, \emph{filter}, and \emph{pipe} in the package
tidyverse (Wickham and RStudio 2017) of R statistical software(\emph{R}
2019). For most questions, results are presented as: (1) percent of
respondents, (2) percent of answers and (3) percent of number of
hectares represented. Not every respondent answered every question; for
some questions, respondents were allowed to select multiple choices.
Moreover, survey respondents were grouped according to their region as
listed in the demographic geopolitical Brazilian map: North, Northeast,
Midwest, Southeast, and South (Figure 1).

\hypertarget{results-and-discussion}{%
\section{Results and Discussion}\label{results-and-discussion}}

\hypertarget{demographics}{%
\subsection{Demographics}\label{demographics}}

Survey answers were obtained from 343 stakeholders, representing 21 of
27 Brazilian states. Most survey respondents were located in the South
(43\%) and Southeast (38\%) regions of Brazil (Table 1A); however 43\%
of managed ha represented in the survey are in the Midwest region (Table
1B). The South and Southeast regions are small/medium farm size
(\textless500 ha), while the Midwest, North and Northeast regions
represent the newly expanded agricultural region in Brazil, with farm
size of \textgreater{} 500 ha up to 100000 ha (Dias et al. 2016). Most
survey respondents identified themselves as agronomists (69\%), followed
by university and industry representatives (22\%), growers (21\%), and
consultants (9\%, Table 1C). Also, respondents represent a total of 5,7
million crop ha, a representative area as there are 78 million ha of
Brazilian territory occupied with crops and planted forest (IBGE 2019).

\hypertarget{crop-management}{%
\subsection{Crop Management}\label{crop-management}}

The survey showed that only 16\% of respondents manage crops in
conventional tillage in Brazil, with highest no-till practice in the
Midwest (71\%) (Table 2A). Six out 10 respondents adopt/recommend cover
crops to some extent (Table 2B), with \emph{Avena sativa} L. (48\%),
\emph{Crotalaria juncea} L. (27\%), \emph{Pennisetum glaucum} (29\%),
\emph{Urochloa} spp. (27\%), \emph{Lolium multiflorum} L. (22\%), and
\emph{Brassica rapa} L. (16\%) ranked as the top cover crop species
adopted by respondents (Table 2C). Moreover, crop-livestock integration
is adopted by 37\% of respondents in Brazil (Table 2D). Survey results
show that crop succession is a common practice in Brazil, 71\% of
respondents manage at least two crops in the same land within a year
(Table 2E). In the South, nearly 40\% of respondents grow three crops in
the same land within a year but 20\% in the Midwest, which strongly rely
on two crop succession systems (74\%). Soybean is usually planted as the
first crop, followed by corn or cotton, and pulse, winter or cover crops
(Cerdeira et al. 2011). The number of crops per year is likely a result
of moisture availability due to regular rainfall in the Southern than
Northern Brazilian states (Alvares et al. 2013). Moreover, 50\% and 32\%
of survey respondents managed rainfed, and partially irrigated fields,
respectively (Table 2F). Therefore, predominant practices in Brazil
include the adoption of no-till, crop succession, and cover crops.

No-till strongly contributed for the expansion of annual crops in
Brazil, especially in the \emph{Cerrado} (savanna biome) area in the
1980s onwards (Sanders and Bein 1976). The geography of \emph{Cerrado}
biome includes the Midwest and parts of the Southeast, North and
Northeast region of Brazil (Figure 1). The \emph{Cerrado} is
characterized with favorable topography for agriculture but low soil
fertility (Goedert 1983), which was mainly used for pastures. In the
early 2000s, it was estimated that 11\% and 41\% of the \emph{Cerrado}
the area was covered with cropland and planted pastures, respectively
(Klink and Machado 2005). The cropland in the \emph{Cerrado} expanded
81\% from 2000 to 2014, mainly replacing poorly managed pastures (Zalles
et al. 2019). No-till, crop succession, cover crops, and crop-livestock
integration strategies have resulted in increased soil chemical and
physical properties in Brazil, especially in the \emph{Cerrado} biome
(de Moraes et al. 2014, Yamada 2005).

The current expansion of cropland is occuring to the new \emph{Cerrado}
areas in the states of Maranhão, Tocantins, Piauí, Bahia (MA-TO-PI-BA)
and in the Northeast and North parts of the Amazonian biome in the state
of Pará (North region) (Lucio et al. 2019, Zalles et al. 2019). The
steady increase of cropland in Brazil is partially due to the success of
no-till and soybean production in the \emph{Cerrado} (de Araújo et al.
2019, Fearnside 2001). The importance of soybean for the Brazilian
agriculture is highlighted in our survey as it is the most managed crop
across the five major regions (Figure 2). Currently, there are 36
million ha of soybean grown with productivity of 119 million tonnes of
grain, which makes soybean the top agricultural export commodity of
Brazil (Oliveira 2016).

\hypertarget{weed-management}{%
\subsection{Weed Management}\label{weed-management}}

\hypertarget{herbicide-programs}{%
\subsubsection{Herbicide Programs}\label{herbicide-programs}}

The wide adoption of no-till soybean systems in Brazil would be less
likely without glyphosate. Because glyphosate is a non-selective and
systemic herbicide, it provides high vegetation control (Duke and Powles
2008). Over 80\% of respondents spray/manage burndown herbicides prior
to annual crops establishment in Brazil (Table 3). High glyphosate
reliance is clearly demonstrated as this is the main herbicide used for
burndown weed control in several annual and perennial cropping-systems
(Figure 3). The synthetic-auxin (e.g.~2,4-D), photosystem I (PSI)-
(e.g., paraquat) and protoporphyrinogen oxidase (PPO)- (e.g.,
saflufenacil) inhibitor herbicides are additional herbicide options
sprayed as part of burndown programs. Burndown glyphosate application is
also commonly used to terminate cover crop species, no-till crop
establishment. The survey also showed glyphosate as a foundation for
POST-emergence weed management in corn, cotton and soybean (Figure 3).
For instance, it has been documented that within a soybean season,
glyphosate is typically sprayed three times in Rio Grande do Sul state
(RS, South region) (da Rosa Ulguim et al. 2017).

The use of PRE herbicides is not popular as burndown and POST herbicide
programs (Table 3). PRE herbicides are costly, restricted due to crop
succession (Reis et al. 1BC--2018) and typically not adopted in absence
of HR weeds. In addition, cover crop residue from burndown applications
result in a physical barrier that may either prevent germination of
early-season weed species (Altieri et al. 2011) or prevent sprayed PRE
herbicide reaching the soil (Christoffoleti et al. 2007), reducing PRE
herbicide efficacy on weeds.

\hypertarget{troublesome-and-gr-weeds}{%
\subsubsection{Troublesome and GR
weeds}\label{troublesome-and-gr-weeds}}

Survey results indicate that the top five problematic weed species in
Brazil are glyphosate-tolerant (\emph{Ipomoea} spp. and \emph{Commelina}
spp.) and GR (\emph{Conyza} spp., \emph{D. insularis} and \emph{E.
indica}) (Figure 4). Distribution of troublesome (Figure 4) and GR weeds
(Table 4A) varied across regions. Although ranked amongst the most
problematic grass weed because of high capacity to evolve resistance to
herbicides (Preston et al. 2009), \emph{Lolium multiflorum} L., a
cool-season grass, is mainly adapted to the South region of Brazil
(Table 4B and Figure 4) (Lucio et al. 2019). However, \emph{Conyza} spp,
an annual species adapted to no-till areas {[}Lucio et al. (2019); {]},
is reported as the most widespread weed species presented in nearly 50\%
of soybeans cropland of Brazil (Lucio et al. 2019). Because of its
intrinsic biology, \emph{Conyza} spp. seeds may reach the planetary
boundary layer (140 m) reaching 500 km seed dispersal (Shields et al.
2006), which strongly contributes for the spreading of \emph{Conyza}
spp. to adjacent and non-adjacent areas (Dauer et al. 2007). The
seed-mediated flow also plays an important role in distribution of other
herbicide resistance weeds. The first report of GR-\emph{A. palmeri} and
GR-\emph{D. insularis} in South America were in Brazilian neighboring
countries Argentina and Paraguay, respectively (Heap 2020). It is
hypothesized that seeds from these two weed species migrated to Brazil
through equipment, human traffic, and/or animals. For example,
GR-\emph{D. insularis} is widespread across Brazilian regions but was
first reported in 2016 western Paraná (PR, south of Brazil) near
Paraguay (Ovejero et al. 2017). Genetic similarities within GR-\emph{D.
insularis} biotypes from Paraguay and Paraná were found but not with GR
\emph{D. insularis} biotypes from Southeast and Midwest (Takano et al.
2018), suggesting that evolution of GR-\emph{D. insularis} is occurring
through seed-mediated flow and independent selection.

Although 83\% of survey respondents are satisfied with their current
levels of weed control (good or excellent;Table 4F), GR weeds have been
documented in orchard, cereal, legume (Vila‐Aiub et al. 2008), and are
on the rise across Brazilian cropping-systems. Eight weed species have
evolved resistance to glyphosate in Brazil, including four monocots and
four dicots (Brunharo et al. 2016, Heap 2020, Küpper et al. 2017, Takano
et al. 2019). Recent reports have documented glyphosate failure to
control \emph{A. hybridus} (HRAC-BR 2019) and \emph{Echinochloa colona}
(Pivetta et al. 2018) in Brazil. Other HR, including ALS-inhibitor
herbicides are widespread in Brazil. For example, ALS-inhibitor
herbicides are the foundation for weed control in rice, wheat, and
soybeans (Figure 4), crops commonly grown in the South, and 87\% of
respondents are managing ALS-resistant weeds in that region (Table 4D).
Weed resistance to ACCase-inhibitor herbicides is also a major problem
in Brazil. The weed species \emph{Urochloa plantaginea}, \emph{Digitaria
ciliaris}, \emph{E. indica}, \emph{Lolium} spp., \emph{A. fatua},
\emph{Echinochloa crus-galli}, and \emph{D. insularis} have evolved
resistance to ACCase-inhibitor herbicides (Heap 2020). The number of
biotypes with multiple HR in increasing in Brazil, including \emph{E.
indica}, \emph{E. crus-galli}, and \emph{D. insularis} with resistance
to ACCase-, ALS-, and/or EPSPS-inhibitors. Moreover, a \emph{Conyza}
spp. biotype was reported resistant to 2,4-D, PSI-, PSII-, PPO-, and
EPSPS- inhibitor herbicides (Heap 2020), which certainly increases the
complexity of weed management in cropping-systems where such biotypes
are present.

\hypertarget{new-technologies-24-d-and-dicamba}{%
\subsubsection{New Technologies: 2,4-D and
Dicamba}\label{new-technologies-24-d-and-dicamba}}

Our survey shows respondents willingness to adopt synthetic AR crops
(57\%) in Brazil (Figure 5). Dicamba and 2,4-D provide effective control
of broadleaf but no grass weed species control. It has been demonstrated
the effective control of dicamba on \emph{Amaranthus} spp. (Schryver et
al. 2017) and 2,4-D on \emph{Conyza} ssp (Frene et al. 2018). Over 90\%
of growers surveyed in Nebraska reported weed management improve after
using DR crops (Werle et al. 2018). However, if adopted, dicamba or
2,4-D would have to be mixed with a graminicide herbicides given main
weed problems in Brazil are GR grasses, such as \emph{D. insularis},
\emph{Eleusine indica}, and \emph{Lolium multiflorum} (Lucio et al.
2019). Studies documented that tank mixing 2,4-D (Li et al. 2020) or
dicamba (Hart and Wax 1996, Underwood et al. 2016) antagonizes grass
weed control. In addition, dicamba with glyphosate reduces pH, resulting
in increased dicamba concentration in the air (Mueller and Steckel
2019a, 2019b). Therefore, 2,4-D and dicamba has not benefit managing
grass weed species and which raises the concerns of OTM in Brazil.

The OTM of dicamba or 2,4-D leading to injury in sensitive vegetation is
currently a major issue in the United States (Knezevic et al. 2018,
Kniss 2018, Soltani et al. 2020) Studies documented that dicamba
concentration in the air following application increased with
temperature (Jones et al. 2019, Mueller and Steckel 2019a). In Brazil,
climatic conditions vary from tropical (with or without a dry season) to
subtropical, with annual mean temperatures \textgreater{} 26 C in
agricultural areas (Alvares et al. 2013). In addition. dicamba or 2,4-D
sensitive crops such as grapes, vegetables, orchards, soybean
(\emph{Glycine max} L. Merr), cotton (\emph{Gossypium hirsutum} L.),
common bean (\emph{Phaseolus vulgaris}) are largely grown in Brazil.
Micro-rates of dicamba or 2,4-D may cause visible injury on non-AR
soybeans (Osipitan et al. 2019), grapes (Mohseni-Moghadam et al. 2016),
and tomato (Knezevic et al. 2018). With AR crops, 2,4-D and dicamba
herbicides are likely to be sprayed in large areas, which increases the
chances of OTM onto sensitive vegetation. In Brazil, there is no
published data regarding potential off-target movement of new dicamba or
2,4-D formulations. Further studies are needed to evaluate the impact of
spraying large areas with dicamba and 2,4-D under tropical conditions.
With the introduction of synthetic AR crops in Brazil, spraying dicamba
and 2,4-D may require restrictions and extra herbicide applicator
training. Although it is not required in the country, nearly 70\% of
survey respondents said applicators received some form of training.

\hypertarget{limitations-for-weed-management}{%
\subsection{Limitations for Weed
Management}\label{limitations-for-weed-management}}

As highlighted in our survey, HR weeds are a major constraint for weed
management in Brazil (Table 6A). Although Brazil has fewer documented
cases of HR weeds when compared to Australia, United States, and Canada
(Heap 2020), the upcoming AR crops technologies do not address the
current HR-grass weed problems in the country. Managing HR weeds in
Brazil may require additional adoption of non chemical strategies or
introduction of new herbicide SOA effective on grass species. Brazilian
growers already employ multiple effective non chemical strategies,
including diversity of crops, crop succession in season, no-till, cover
crops, and/or cultural weed management (Table 4E). A new non chemical
weed control strategy, harvest seed weed control (Walsh et al. 2012), is
a valuable tool for minimizing HR weeds but is still neither available
nor evaluated/studied, to our knowledge, in Brazil. Nonetheless, the
evolution of HR grass weed species and absence of new effective
technologies are threatening the sustainability of Brazilian
agricultural production.

Survey respondents reported industry as the main source of information
for crop and weed management in Brazil (Table 6B). Despite being
incredibly valuable, industry information can be biased towards
portfolios. In contrast, sources of unbiased information from basic and
applied research are public institutions, including Universities and
Embrapa (Brazilian Agricultural Research Corporation). Therefore, there
is a need for an increase on collaborative work on basic and applied
research in Brazil due to the upcoming weed herbicide resistance crisis
and introduction of novel and complex to adopt technologies which will
demand research and education for proper adoption.

\hypertarget{conclusion}{%
\section{Conclusion}\label{conclusion}}

The survey results presented herein highlighted the current status and
the difference in cropping-systems and weed management practices adopted
across Brazil. Our survey showed the trends in conservation agricultural
practices and advances the knowledge regarding current weed management
strategies of Brazilian agriculture. Brazilian stakeholders are
progressive in the sense of adopting conservation agricultural practices
and new technologies. However, introduction of new technologies focused
on the United States (e.g., synthetic AR crops) for weed management may
not address the major weed problems in Brazil but potentially generate a
new challenge, OTM of herbicides into sensitive vegetation. Therefore,
we urge that academics, growers, industry and policy makers (1) expand
monitoring herbicide resistance weeds, (2) the increase research on
non-chemical weed management strategies and (3) increase investments on
public databases, surveys, basic and applied research to support
decisions regarding the introduction and adoption of novel agricultural
technologies.

\hypertarget{acknowledgements}{%
\section{Acknowledgements}\label{acknowledgements}}

This research received no specific grant from any funding agency,
commercial or not-for-profit sectors. The authors would like to thank
all respondents for participating in the survey and Dr.~Dênis Mariano
for developing Figure 1.

\hypertarget{conflicts-of-interest}{%
\section{Conflicts of Interest}\label{conflicts-of-interest}}

No conflicts of interest have been declared.

\hypertarget{references}{%
\section{References}\label{references}}

\hypertarget{refs}{}
\leavevmode\hypertarget{ref-altieri_enhancing_2011}{}%
Altieri MA, Lana MA, Bittencourt HV, Kieling AS, Comin JJ, Lovato PE
(2011) Enhancing Crop Productivity via Weed Suppression in Organic
No-Till Cropping Systems in Santa Catarina, Brazil. Journal of
Sustainable Agriculture 35:855--869

\leavevmode\hypertarget{ref-alvares_koppens_2013}{}%
Alvares CA, Stape JL, Sentelhas PC, de Moraes Gonçalves JL, Sparovek G
(2013) Köppen's climate classification map for Brazil. Meteorologische
Zeitschrift 22:771--728

\leavevmode\hypertarget{ref-appleby_history_2005}{}%
Appleby AP (2005) A history of weed control in the United States and
Canada---a sequel. Weed Science 53:762--768

\leavevmode\hypertarget{ref-araujo_spatiotemporal_2019}{}%
de Araújo MLS, Sano EE, Bolfe ÉL, Santos JRN, dos Santos JS, Silva FB
(2019) Spatiotemporal dynamics of soybean crop in the Matopiba region,
Brazil (1990--2015). Land Use Policy 80:57--67

\leavevmode\hypertarget{ref-benbrook_trends_2016}{}%
Benbrook CM (2016) Trends in glyphosate herbicide use in the United
States and globally. Environ Sci Eur 28:3

\leavevmode\hypertarget{ref-bish_survey_2017}{}%
Bish MD, Bradley KW (2017) Survey of Missouri Pesticide Applicator
Practices, Knowledge, and Perceptions. Weed Technology 31:165--177

\leavevmode\hypertarget{ref-bonny_herbicide-tolerant_2011}{}%
Bonny S (2011) Herbicide-tolerant Transgenic Soybean over 15 Years of
Cultivation: Pesticide Use, Weed Resistance, and Some Economic Issues.
The Case of the USA. Sustainability 3:1302--1322

\leavevmode\hypertarget{ref-brunharo_confirmation_2016}{}%
Brunharo CA, Patterson EL, Carrijo DR, Melo MS de, Nicolai M, Gaines TA,
Nissen SJ, Christoffoleti PJ (2016) Confirmation and mechanism of
glyphosate resistance in tall windmill grass (\emph{Chloris}
\emph{Elata}) from Brazil. Pest Management Science 72:1758--1764

\leavevmode\hypertarget{ref-cerdeira_agricultural_2011}{}%
Cerdeira AL, Gazziero DLP, Duke SO, Matallo MB (2011) Agricultural
Impacts of Glyphosate-Resistant Soybean Cultivation in South America.
Journal of Agricultural and Food Chemistry 59:5799--5807

\leavevmode\hypertarget{ref-christoffoleti_conservation_2007}{}%
Christoffoleti PJ, de Carvalho SJP, López-Ovejero RF, Nicolai M, Hidalgo
E, da Silva JE (2007) Conservation of natural resources in Brazilian
agriculture: Implications on weed biology and management. Crop
Protection 26:383--389

\leavevmode\hypertarget{ref-dauer_temporal_2007}{}%
Dauer JT, Mortensen DA, Vangessel MJ (2007) Temporal and spatial
dynamics of long-distance Conyza canadensis seed dispersal: Dynamics of
\emph{Conyza} \emph{Canadensis} seed dispersal. Journal of Applied
Ecology 44:105--114

\leavevmode\hypertarget{ref-dias_patterns_2016}{}%
Dias LCP, Pimenta FM, Santos AB, Costa MH, Ladle RJ (2016) Patterns of
land use, extensification, and intensification of Brazilian agriculture.
Global Change Biology 22:2887--2903

\leavevmode\hypertarget{ref-duke_why_2012}{}%
Duke SO (2012) Why have no new herbicide modes of action appeared in
recent years? Pest Management Science 68:505--512

\leavevmode\hypertarget{ref-duke_glyphosate:_2008}{}%
Duke SO, Powles SB (2008) Glyphosate: A once-in-a-century herbicide.
Pest Management Science 64:319--325

\leavevmode\hypertarget{ref-fearnside_soybean_2001}{}%
Fearnside PM (2001) Soybean cultivation as a threat to the environment
in Brazil. Environmental Conservation 28:23--38

\leavevmode\hypertarget{ref-frene_enlist_2018}{}%
Frene RL, Simpson DM, Buchanan MB, Vega ET, Ravotti ME, Valverde PP
(2018) Enlist E3™ Soybean Sensitivity and Enlist™ Herbicide-Based
Program Control of Sumatran Fleabane (Conyza sumatrensis). Weed
Technology 32:416--423

\leavevmode\hypertarget{ref-gianessi_value_2007}{}%
Gianessi LP, Reigner NP (2007) The Value of Herbicides in U.S. Crop
Production. Weed Technology 21:559--566

\leavevmode\hypertarget{ref-goedert_management_1983}{}%
Goedert WJ (1983) Management of the Cerrado soils of Brazil: A review.
Journal of Soil Science 34:405--428

\leavevmode\hypertarget{ref-green_rise_2018}{}%
Green JM (2018) The rise and future of glyphosate and
glyphosate-resistant crops. Pest Management Science 74:1035--1039

\leavevmode\hypertarget{ref-hager_observations_2018}{}%
Hager A (2018) Observations of the Midwest weed extension scientists.
Page 98 \emph{in} Proceedings of the 72nd Annual Meeting of the North
Central Weed Science Society. Saint Louis, MO

\leavevmode\hypertarget{ref-hart_dicamba_1996}{}%
Hart SE, Wax LM (1996) Dicamba Antagonizes Grass Weed Control with
Imazethapyr by Reducing Foliar Absorption. Weed Technology 10:828--834

\leavevmode\hypertarget{ref-heap_list_2020}{}%
Heap I (2020) List of Herbicide Resistant Weeds by Country.
\url{http://www.weedscience.org/Summary/Country.aspx}. Accessed January
29, 2020

\leavevmode\hypertarget{ref-hrac-br_possible_2019}{}%
HRAC-BR (2019) Possible glyphosate resistance of \emph{Amaranthus}
\emph{Hybridus} reported in Brazil.
\url{http://news.agropages.com/News/Detail-29964.htm}. Accessed April
24, 2019

\leavevmode\hypertarget{ref-ibge_serie_2019}{}%
IBGE (2019) Série histórica da estimativa anual da área plantada, área
colhida, produção e rendimento médio dos produtos das lavouras.
\url{https://sidra.ibge.gov.br/tabela/6588}. Accessed April 16, 2019

\leavevmode\hypertarget{ref-jones_off-target_2019}{}%
Jones GT, Norsworthy JK, Barber T, Gbur E, Kruger GR (2019) Off-target
Movement of DGA and BAPMA Dicamba to Sensitive Soybean. Weed Technology
33:51--65

\leavevmode\hypertarget{ref-j._t._holstun_weed_1960}{}%
J. T. Holstun J, O. B. Wooten J, McWhorter CG, Crowe GB (1960) Weed
Control Practices, Labor Requirements and Costs in Cotton Production.
Weeds 8:232--243

\leavevmode\hypertarget{ref-klink_conservation_2005}{}%
Klink CA, Machado RB (2005) Conservation of the Brazilian Cerrado.
Conservation Biology 19:707--713

\leavevmode\hypertarget{ref-knezevic_sensitivity_2018}{}%
Knezevic SZ, Osipitan OA, Scott JE (2018) Sensitivity of Grape and
Tomato to Micro-rates of Dicamba-based Herbicides. Journal of
Horticulture 05

\leavevmode\hypertarget{ref-kniss_soybean_2018}{}%
Kniss AR (2018) Soybean Response to Dicamba: A Meta-Analysis. Weed
Technology 32:507--512

\leavevmode\hypertarget{ref-kupper_multiple_2017}{}%
Küpper A, Borgato EA, Patterson EL, Netto AG, Nicolai M, Carvalho SJP
de, Nissen SJ, Gaines TA, Christoffoleti PJ (2017) Multiple Resistance
to Glyphosate and Acetolactate Synthase Inhibitors in Palmer Amaranth
(\emph{Amaranthus} \emph{Palmeri}) Identified in Brazil. Weed Science
65:317--326

\leavevmode\hypertarget{ref-li_24-d_2020}{}%
Li J, Han H, Bai L, Yu Q (2020) 2,4-D Antagonizes glyphosate in
glyphosate-resistant barnyard grass \emph{Echinochloa} \emph{Colona}. J
Pestic Sci

\leavevmode\hypertarget{ref-lucio_dispersal_2019}{}%
Lucio FR, Kalsing A, Adegas FS, Rossi CVS, Correia NM, Gazziero DLP,
Silva AF da (2019) Dispersal and Frequency of Glyphosate-Resistant and
Glyphosate-Tolerant Weeds in Soybean-producing Edaphoclimatic
Microregions in Brazil. Weed Technology 33:217--231

\leavevmode\hypertarget{ref-marth_24-dichlorophenoxyacetic_1944}{}%
Marth PC, Mitchell JW (1944) 2,4-Dichlorophenoxyacetic Acid as a
Differential Herbicide. Botanical Gazette 106:224--232

\leavevmode\hypertarget{ref-mcfarland_triazine_2011}{}%
McFarland J, Burnside O (2011) The Triazine Herbicides. Elsevier. 604 p

\leavevmode\hypertarget{ref-mohseni-moghadam_response_2016}{}%
Mohseni-Moghadam M, Wolfe S, Dami I, Doohan D (2016) Response of Wine
Grape Cultivars to Simulated Drift Rates of 2,4-D, Dicamba, and
Glyphosate, and 2,4-D or Dicamba Plus Glyphosate. Weed Technology
30:807--814

\leavevmode\hypertarget{ref-de_moraes_integrated_2014}{}%
de Moraes A, Carvalho PC de F, Anghinoni I, Lustosa SBC, Costa SEVG de
A, Kunrath TR (2014) Integrated crop--livestock systems in the Brazilian
subtropics. European Journal of Agronomy 57:4--9

\leavevmode\hypertarget{ref-mueller_dicamba_2019}{}%
Mueller TC, Steckel LE (2019a) Dicamba volatility in humidomes as
affected by temperature and herbicide treatment. Weed Technol
33:541--546

\leavevmode\hypertarget{ref-mueller_spray_2019}{}%
Mueller TC, Steckel LE (2019b) Spray mixture pH as affected by dicamba,
glyphosate, and spray additives. Weed Technol 33:547--554

\leavevmode\hypertarget{ref-oliveira_geopolitics_2016}{}%
Oliveira G de LT (2016) The geopolitics of Brazilian soybeans. The
Journal of Peasant Studies 43:348--372

\leavevmode\hypertarget{ref-osipitan_glyphosate-resistant_2019}{}%
Osipitan OA, Scott JE, Knezevic SZ (2019) Glyphosate-Resistant Soybean
Response to Micro-Rates of Three Dicamba-Based Herbicides. Agrosystems
2:1--8

\leavevmode\hypertarget{ref-ovejero_frequency_2017}{}%
Ovejero RFL, Takano HK, Nicolai M, Ferreira A, Melo MSC, Cavenaghi AL,
Christoffoleti PJ, Oliveira RS (2017) Frequency and Dispersal of
Glyphosate-Resistant Sourgrass (\emph{Digitaria} \emph{Insularis})
Populations across Brazilian Agricultural Production Areas. Weed Science
65:285--294

\leavevmode\hypertarget{ref-owen_weed_2008}{}%
Owen MD (2008) Weed species shifts in glyphosate-resistant crops. Pest
Management Science 64:377--387

\leavevmode\hypertarget{ref-owen_current_2000}{}%
Owen MDK (2000) Current use of transgenic herbicide-resistant soybean
and corn in the USA. Crop Protection 19:765--771

\leavevmode\hypertarget{ref-padgette_development_1995}{}%
Padgette SR, Kolacz KH, Delannay X, Re DB, LaVallee BJ, Tinius CN,
Rhodes WK, Otero YI, Barry GF, Eichholtz DA, Peschke VM, Nida DL, Taylor
NB, Kishore GM (1995--1AD) Development, Identification, and
Characterization of a Glyphosate-Tolerant Soybean Line. Crop Science
35:1451--1461

\leavevmode\hypertarget{ref-pivetta_capim_2018}{}%
Pivetta M, Dornelles S, Sanchonete D, Goergen A, Soares J, Brun A,
Pedrollo N (2018) Capim arroz resistente ao glifosato em área de
produção de soja no Rio Grande do Sul. Page 560 \emph{in} Resumos do
XXXI Congresso Brasileiro da Ciência das Plantas Daninhas. Rio de
Janeiro, RJ

\leavevmode\hypertarget{ref-preston_decade_2009}{}%
Preston C, Wakelin AM, Dolman FC, Bostamam Y, Boutsalis P (2009) A
Decade of Glyphosate-Resistant Lolium around the World: Mechanisms,
Genes, Fitness, and Agronomic Management. Weed Science 57:435--441

\leavevmode\hypertarget{ref-r_core_team_r:_2019}{}%
R: A Language and Environment for Statistical Computing (2019). Vienna,
Austria: R Foundation for Statistical Computing

\leavevmode\hypertarget{ref-reis_carryover_2018}{}%
Reis MR, Aquino LÂ, Melo CAD, Silva DV, Dias RC, Reis MR, Aquino LÂ,
Melo CAD, Silva DV, Dias RC (1BC--2018) Carryover of tembotrione and
atrazine affects yield and quality of potato tubers. Acta Scientiarum
Agronomy 40

\leavevmode\hypertarget{ref-da_rosa_ulguim_agronomic_2017}{}%
da Rosa Ulguim A, Agostinetto D, Vargas L, Dias Gomes da Silva J, Moncks
da Silva B, da Rosa Westendorff N (2017) Agronomic factors involved in
low-level wild poinsettia resistance to glyphosate. Revista Brasileira
de Ciências Agrárias 12

\leavevmode\hypertarget{ref-sanders_agricultural_1976}{}%
Sanders JH, Bein FL (1976) Agricultural Development on the Brazilian
Frontier: Southern Mato Grosso. Economic Development and Cultural Change
24:593--610

\leavevmode\hypertarget{ref-schryver_control_2017}{}%
Schryver M, Soltani N, Hooker DC, Robinson DE, Tranel P, Sikkema PH
(2017) Control of glyphosate-resistant common waterhemp (Amaranthus
tuberculatus var rudis) with dicamba and dimethenamid-P in Ontario. Can
J Plant Sci:CJPS--2017--0052

\leavevmode\hypertarget{ref-schwartz-lazaro_midsouthern_2018}{}%
Schwartz-Lazaro LM, Norsworthy JK, Steckel LE, Stephenson DO, Bish MD,
Bradley KW, Bond JA (2018) A Midsouthern Consultant's Survey on Weed
Management Practices in Soybean. Weed Technology 32:116--125

\leavevmode\hypertarget{ref-scursoni_weed_2019}{}%
Scursoni JA, Vera ACD, Oreja FH, Kruk BC, Fuente EB de la (2019) Weed
management practices in Argentina crops. Weed Technology 33:459--463

\leavevmode\hypertarget{ref-shields_horseweed_2006}{}%
Shields EJ, Dauer JT, VanGessel MJ, Neumann G (2006) Horseweed
(\emph{Conyza} \emph{Canadensis}) seed collected in the planetary
boundary layer. Weed Science 54:1063--1067

\leavevmode\hypertarget{ref-sidra-ibge_sistema_2020}{}%
SIDRA-IBGE (2020) Sistema IBGE de Recuperação Automática - SIDRA.
\url{https://sidra.ibge.gov.br/home/}. Accessed March 16, 2020

\leavevmode\hypertarget{ref-soltani_off-target_2020}{}%
Soltani N, Oliveira MC, Alves GS, Werle R, Norsworthy JK, Sprague CL,
Young BG, Reynolds DB, Brown A, Sikkema PH (2020) Off-target movement
assessment of dicamba in North America. Weed Technol:1--13

\leavevmode\hypertarget{ref-takano_proline-106_2019}{}%
Takano HK, Mendes RR, Scoz LB, Ovejero RFL, Constantin J, Gaines TA,
Westra P, Dayan FE, Oliveira RS (2019) Proline-106 EPSPS Mutation
Imparting Glyphosate Resistance in Goosegrass (\emph{Eleusine}
\emph{Indica}) Emerges in South America. Weed Science 67:48--56

\leavevmode\hypertarget{ref-takano_spread_2018}{}%
Takano HK, Oliveira RS de, Constantin J, Mangolim CA, Machado M de FPS,
Bevilaqua MRR (2018) Spread of glyphosate-resistant sourgrass
(\emph{Digitaria} \emph{Insularis}): Independent selections or merely
propagule dissemination? Weed Biology and Management 18:50--59

\leavevmode\hypertarget{ref-troyer_beginning:_2001}{}%
Troyer JR (2001) In the beginning: The multiple discovery of the first
hormone herbicides. Weed Science 49:290--297

\leavevmode\hypertarget{ref-underwood_addition_2016}{}%
Underwood MG, Soltani N, Hooker DC, Robinson DE, Vink JP, Swanton CJ,
Sikkema PH (2016) The Addition of Dicamba to POST Applications of
Quizalofop-p-ethyl or Clethodim Antagonizes Volunteer
Glyphosate-Resistant Corn Control in Dicamba-Resistant Soybean. Weed
Technology 30:639--647

\leavevmode\hypertarget{ref-usda-nass_national_2019-1}{}%
USDA-NASS (2019) National Agricultural Statistics Service - Charts and
Maps - County Maps.
\url{https://www.nass.usda.gov/Charts_and_Maps/Crops_County/\#cr}.
Accessed October 10, 2019

\leavevmode\hypertarget{ref-vilaaiub_glyphosate-resistant_2008-1}{}%
Vila‐Aiub MM, Vidal RA, Balbi MC, Gundel PE, Trucco F, Ghersa CM (2008)
Glyphosate-resistant weeds of South American cropping systems: An
overview. Pest Management Science 64:366--371

\leavevmode\hypertarget{ref-walsh_harrington_2012}{}%
Walsh MJ, Harrington RB, Powles SB (2012) Harrington Seed Destructor: A
New Nonchemical Weed Control Tool for Global Grain Crops. Crop Science
52:1343

\leavevmode\hypertarget{ref-warren_spectacular_1998}{}%
Warren GF (1998) Spectacular Increases in Crop Yields in the United
States in the Twentieth Century. Weed Technology 12:752--760

\leavevmode\hypertarget{ref-werle_survey_2018}{}%
Werle R, Oliveira MC, Jhala AJ, Proctor CA, Rees J, Klein R (2018)
Survey of Nebraska Farmers' Adoption of Dicamba-Resistant Soybean
Technology and Dicamba Off-Target Movement. Weed Technol 32:754--761

\leavevmode\hypertarget{ref-wickham_tidyverse:_2017}{}%
Wickham H, RStudio (2017) Tidyverse: Easily Install and Load the
'Tidyverse'

\leavevmode\hypertarget{ref-wright_robust_2010}{}%
Wright TR, Shan G, Walsh TA, Lira JM, Cui C, Song P, Zhuang M, Arnold
NL, Lin G, Yau K, Russell SM, Cicchillo RM, Peterson MA, Simpson DM,
Zhou N, Ponsamuel J, Zhang Z (2010) Robust crop resistance to broadleaf
and grass herbicides provided by aryloxyalkanoate dioxygenase
transgenes. PNAS 107:20240--20245

\leavevmode\hypertarget{ref-yamada_cerrado_2005}{}%
Yamada T (2005) The Cerrado of Brazil: A Success Story of Production on
Acid Soils. Soil Science and Plant Nutrition 51:617--620

\leavevmode\hypertarget{ref-zalles_near_2019}{}%
Zalles V, Hansen MC, Potapov PV, Stehman SV, Tyukavina A, Pickens A,
Song X-P, Adusei B, Okpa C, Aguilar R, John N, Chavez S (2019) Near
doubling of Brazil's intensive row crop area since 2000. Proceedings of
the National Academy of Sciences 116:428--435

\eleft

\clearpage

\listoftables

\newpage

\begin{table}[ht!]
\centering
\caption{Respondents demographics of the 2018 Cropping systems weed management survey.}
\label{tab:my-table}
\begin{tabular}{@{}lllllll@{}}
\toprule
                            & \multicolumn{6}{c}{Region}                               \\ \cmidrule(l){2-7} 
Demographics                & Brazil & North & Northeast & Midwest & Southeast & South \\ \hline
                            & \multicolumn{6}{c}{--------\%--------}                           \\ 
A. Respondents (n=279)      &        & 4     & 8         & 23      & 38        & 43    \\
B. Hectares managed (n=123) &        & 9     & 6         & 43      & 41        & 14    \\
\hspace{6mm} Hectares (ha)       &   5.7 mi     &       &           &         &           &       \\
C. Role (n=277)             &        &       &           &         &           &       \\
\hspace{3mm}\textit{Agronomist}         & 69     & 68    & 68        & 68      & 73        & 61    \\
\hspace{3mm}\textit{Consultant}         & 9      & 0     & 14        & 16      & 13        & 3     \\
\hspace{3mm}\textit{Industry}           & 22     & 45    & 36        & 35      & 26        & 13    \\
\hspace{3mm}\textit{Grower}             & 21     & 18    & 5         & 22      & 24        & 17    \\
\hspace{3mm}\textit{University}         & 22     & 18    & 18        & 11      & 26        & 22    \\
\hspace{3mm}\textit{Other}              & 7      & 7     & 8         & 3       & 4         & 11    \\ \bottomrule
\end{tabular}
\end{table}

\newpage

\begin{table}[ht!]
\centering
\caption{Cropping-system managament strategies adopted in Brazil according to the 2018 cropping systems weed management survey.}
\label{tab:my-table}
\begin{tabular}{@{}lllllll@{}}
\toprule
                                  & \multicolumn{6}{c}{Region}                               \\ \cmidrule(l){2-7} 
Cropping systems                  & Brazil & North & Northeast & Midwest & Southeast & South \\ \hline 
                                  & \multicolumn{6}{c}{--------\%--------}                             \\ 
A. Conservation tillage (no-till) &        &       &           &         &           &       \\
\hspace{3mm}\textit{Yes}                      & 61     & 55    & 50        & 71      & 51        & 67    \\
\hspace{3mm}\textit{Partially}                & 22     & 27    & 18        & 18      & 18        & 27    \\
\hspace{3mm}\textit{No}                       & 16     & 18    & 32        & 11      & 31        & 6     \\
\hspace{3mm}\textit{n}                        & 273    & 11    & 22        & 63      & 99        & 119   \\
B. Cover crop                     &        &       &           &         &           &       \\
\hspace{3mm}\textit{Yes}                      & 61     & 55    & 55        & 58      & 52        & 68    \\
\hspace{3mm}\textit{No}                       & 39     & 45    & 45        & 42      & 48        & 32    \\
\hspace{3mm}\textit{n}                        & 273    & 11    & 22        & 7       & 99        & 119   \\
C. Cover crop species             &        &       &           &         &           &       \\
\hspace{3mm}\textit{Avena sativa L.}               & 48     & 0     & 5         & 7       & 50        & 50    \\
\hspace{3mm}\textit{Brassica rapa}            & 16     & 0     & 5         & 7       & 14        & 17    \\
\hspace{3mm}\textit{Crotalaria juncea L.}                & 27     & 0     & 14        & 28      & 91        & 5     \\
\hspace{3mm}\textit{Lolium multiflorum L.}         & 22     & 0     & 0         & 2       & 9         & 27    \\
\hspace{3mm}\textit{Pennisetum glaucum}             & 29     & 27    & 27        & 28      & 68        & 6     \\
\hspace{3mm}\textit{Urochloa spp.}               & 27     & 27    & 27        & 37      & 68        & 4     \\
\hspace{3mm}\textit{Other}                    & 3      & 1     & 2         & 1       & 7         & 2     \\
\hspace{3mm}\textit{n}                        & 143    & 11    & 22        & 57      & 22        & 113   \\
D. Crop-livestock integration     &        &       &           &         &           &       \\
\hspace{3mm}\textit{Yes}                      & 37     & 45    & 14        & 46      & 22        & 48    \\
\hspace{3mm}\textit{No}                       & 63     & 55    & 86        & 54      & 78        & 52    \\
\hspace{3mm}\textit{n}                        & 256    & 11    & 22        & 57      & 93        & 114   \\ 
E. Crop succession                &        &       &           &         &           &       \\
\hspace{3mm}\textit{1}                        & 29     & 9     & 36        & 7       & 27        & 34    \\
\hspace{3mm}\textit{2}                        & 41     & 64    & 41        & 74      & 43        & 29    \\
\hspace{3mm}\textit{3}                        & 30     & 27    & 23        & 20      & 29        & 37    \\
\hspace{3mm}\textit{n}                        & 271    & 11    & 22        & 61      & 99        & 119   \\
F. Irrigation                     &        &       &           &         &           &       \\
\hspace{3mm}\textit{Yes}                      & 18     & 9     & 14        & 6       & 17        & 24    \\
\hspace{3mm}\textit{Partially}                & 32     & 9     & 41        & 33      & 34        & 29    \\
\hspace{3mm}\textit{No}                       & 50     & 82    & 45        & 60      & 48        & 47    \\
\hspace{3mm}\textit{n}                        & 272    & 11    & 22        & 63      & 99        & 119   \\ \bottomrule
\end{tabular}
\end{table}

\newpage

\begin{table}[ht!]
\centering
\caption{Herbicide program for weed managament in multiple crops in Brazil according to the 2018 Cropping systems weed management survey.}
\label{tab:my-table}
\begin{tabular}{lllll}
\hline
\multirow{2}{*}{Crops} & \multicolumn{4}{c}{Weed Management Program}             \\ \cline{2-5} 
                       & Burndown & PRE & POST & Harvest aid \\ \hline
                       & \multicolumn{4}{c}{--------\%--------}                  \\
Corn (n=119)           & 85       & 41            & 92             & -           \\
Cotton (n=23)          & 87       & 70            & 87             & 39          \\
Coffee (n=20)          & 35       & 25            & 85             & -           \\
Citrus (n=19)          & 32       & 16            & 68             & -           \\
Eucaliptus (n=15)      & 80       & 47            & 67             & -           \\
Rice (n=45)            & 91       & 76            & 93             & -           \\
Common bean (n=57)     & 93       & 44            & 81             & 58          \\
Sorghum (n=22)         & 100      & 55            & 68             & -           \\
Soybean (n=159)        & 82       & 53            & 81             & 61          \\
Sugarcane (n=31)       & 71       & 87            & 77             & -           \\
Vegetables (n=16)      & 69       & 50            & 69             & -           \\
Wheat (n=33)           & 100      & 33            & 94             & 30          \\
Winter crops (n=30)    & 97       & 20            & 70             & -           \\ \hline
\end{tabular}
\end{table}

\newpage

\begin{table}[ht!]
\centering
\caption{Weed managament strategies in Brazil according to the 2018 Cropping systems weed management survey.}
\label{tab:my-table}
\begin{tabular}{@{}lllllll@{}}
\toprule
                                       & \multicolumn{6}{c}{Region}                               \\ \cmidrule(l){2-7} 
Weed Management                        & Brazil & North & Northeast & Midwest & Southeast & South \\ \hline
                                       & \multicolumn{6}{c}{--------\%--------}                             \\ 
A. Glyphosate resistance               &        &       &           &         &           &       \\
\hspace{3mm}\textit{Yes}                           & 74     & 73    & 64        & 79      & 73        & 80    \\
\hspace{3mm}\textit{Not sure}                      & 12     & 18    & 27        & 12      & 13        & 7     \\
\hspace{3mm}\textit{No}                            & 14     & 9     & 9         & 9       & 14        & 13    \\
\hspace{3mm}\textit{n}                             & 258    & 11    & 22        & 57      & 94        & 114   \\
B. Glyphosate resistant weeds          &        &       &           &         &           &       \\
\hspace{3mm}\textit{Amaranthus palmeri}            & 2      & 0     & 4         & 4       & 1         & 0     \\
\hspace{3mm}\textit{Choris elata}                  & 7      & 13    & 7         & 7       & 6         & 9     \\
\hspace{3mm}\textit{Conyza spp.}                   & 82     & 88    & 79        & 71      & 79        & 91    \\
\hspace{3mm}\textit{Digitaria insularis}           & 56     & 75    & 79        & 91      & 82        & 25    \\
\hspace{3mm}\textit{Eleusine indica}               & 31     & 50    & 43        & 44      & 28        & 25    \\
\hspace{3mm}\textit{Lolium multiflorum L.}                   & 28     & 13    & 14        & 9       & 10        & 54    \\
\hspace{3mm}\textit{n}                             & 190    & 8     & 14        & 45      & 67        & 91    \\
C. Other herbicide resistance &        &       &           &         &           &       \\
\hspace{3mm}\textit{Yes}                           & 46     & 55    & 50        & 79      & 37        & 54    \\
\hspace{3mm}\textit{Not sure}                               & 24     & 36    & 23        & 12      & 29        & 19    \\
\hspace{3mm}\textit{No}                            & 30     & 9     & 27        & 9       & 34        & 26    \\
\hspace{3mm}\textit{n}                             & 257    & 11    & 22        & 57      & 94        & 114   \\
D. Herbicide resistance SOA   &        &       &           &         &           &       \\
\hspace{3mm}\textit{ALS inhibitor}                & 78     & 50    & 63        & 76      & 73        & 87    \\
\hspace{3mm}\textit{ACCase inhibitor}              & 31     & 50    & 63        & 44      & 46        & 21    \\
\hspace{3mm}\textit{HPPD inhibitor}                & 7      & 25    & 38        & 0       & 12        & 2     \\
\hspace{3mm}\textit{PSI inhibitor}                 & 13     & 0     & 38        & 8       & 23        & 10    \\
\hspace{3mm}\textit{PSII inhibitor}                & 12     & 25    & 38        & 16      & 19        & 6     \\
\hspace{3mm}\textit{PPO inhibitor}                 & 11     & 50    & 13        & 16      & 19        & 10    \\
\hspace{3mm}\textit{Synthetic auxin}               & 13     & 25    & 13        & 0       & 8         & 2     \\
\hspace{3mm}\textit{LCFA inhibitor}                & 4      & 25    & 13        & 0       & 8         & 2     \\
\hspace{3mm}\textit{n}                             & 97     & 4     & 8         & 25      & 26        & 52    \\
E. Alternative weed control            &        &       &           &         &           &       \\
\hspace{3mm}\textit{Biological}                    & 5      & 0     & 6         & 2       & 3         & 6     \\
\hspace{3mm}\textit{Cultural}                      & 71     & 100   & 33        & 77      & 65        & 76    \\
\hspace{3mm}\textit{Mechanical}                    & 45     & 25    & 50        & 33      & 58        & 41    \\
\hspace{3mm}\textit{Physical}                               & 15     & 13    & 0         & 2       & 12        & 24    \\
\hspace{3mm}\textit{None}                                   & 15     & 0     & 31        & 14      & 15        & 14    \\
\hspace{3mm}\textit{n}                                      & 192    & 8     & 16        & 43      & 66        & 87    \\
F. Level of weed control               &        &       &           &         &           &       \\
\hspace{3mm}\textit{Excelent}                               & 12     & 44    & 19        & 29      & 17        & 4     \\
\hspace{3mm}\textit{Good}                                  & 71     & 56    & 63        & 64      & 70        & 71    \\
\hspace{3mm}\textit{Low}                                   & 18     & 0     & 19        & 7       & 13        & 24    \\ \bottomrule
\end{tabular}
\end{table}

\newpage

\begin{table}[]
\centering
\caption{Herbicide application technology in Brazil according to the 2018 Cropping systems weed management survey.}
\label{tab:my-table}
\begin{tabular}{@{}lllllll@{}}
\toprule
\multirow{2}{*}{Herbicide application} & \multicolumn{6}{c}{Region}                               \\ \cmidrule(l){2-7} 
                                       & Brazil & North & Northeast & Midwest & Southeast & South \\ \midrule
A. Responsible for application         & \multicolumn{6}{c}{--------\%--------}                             \\
\hspace{3mm}\textit{Ag technician}                 & 17     & 22    & 24        & 20      & 20        & 19    \\
\hspace{3mm}\textit{Agronomist}                    & 30     & 67    & 47        & 30      & 40        & 19    \\
\hspace{3mm}\textit{Applicator specialist}         & 21     & 44    & 18        & 32      & 24        & 15    \\
\hspace{3mm}\textit{Co-op}                         & 3      & 0     & 0         & 0       & 4         & 4     \\
\hspace{3mm}\textit{Grower}                        & 50     & 56    & 41        & 32      & 36        & 71    \\
\hspace{3mm}\textit{Farm employees}                & 50     & 33    & 29        & 68      & 56        & 38    \\
\hspace{3mm}\textit{n}                             & 202    & 9     & 17        & 44      & 70        & 91    \\
B. Herbicide application training      &        &       &           &         &           &       \\
\hspace{3mm}\textit{Yes}                           & 69     & 89    & 64        & 84      & 81        & 56    \\
\hspace{3mm}\textit{Not sure}                      & 16     & 11    & 18        & 7       & 10        & 21    \\
\hspace{3mm}\textit{No}                            & 15     & 0     & 18        & 9       & 9         & 23    \\
\hspace{3mm}\textit{n}                             & 202    & 9     & 17        & 44      & 70        & 91    \\ \bottomrule
\end{tabular}
\end{table}

\pagebreak
\newpage

\begin{table}[ht]
\centering
\caption{General questions regarding weed management strategies in Brazil according to the 2018 Cropping systems weed management survey}
\label{tab:my-table}
\begin{tabular}{@{}lllllll@{}}
\toprule
\multirow{2}{*}{General questions} & \multicolumn{6}{c}{Region}                               \\ \cmidrule(l){2-7} 
                                   & Brazil & North & Northeast & Midwest & Southeast & South \\ \midrule
A. Limitations                     & \multicolumn{6}{c}{--------\%--------}                             \\
\hspace{3mm}\textit{Costs}                     & 53     & 63    & 81        & 32      & 56        & 47    \\
\hspace{3mm}\textit{Limited herbicide options} & 38     & 0     & 35        & 21      & 34        & 38    \\
\hspace{3mm}\textit{Labor}                     & 18     & 0     & 13        & 7       & 15        & 20    \\
\hspace{3mm}\textit{Legislation}               & 30     & 25    & 13        & 13      & 29        & 31    \\
\hspace{3mm}\textit{Weed resistance}           & 69     & 75    & 56        & 38      & 59        & 78    \\
\hspace{3mm}\textit{n}                         & 198    & 8     & 16        & 90      & 68        & 90    \\
B. Source of information           &        &       &           &         &           &       \\
\hspace{3mm}\textit{Consultant}                & 30     & 44    & 25        & 38      & 35        & 24    \\
\hspace{3mm}\textit{Embrapa}                   & 43     & 11    & 31        & 41      & 42        & 49    \\
\hspace{3mm}\textit{Industry}                  & 54     & 78    & 31        & 55      & 65        & 50    \\
\hspace{3mm}\textit{University}                & 52     & 44    & 56        & 48      & 52        & 56    \\
\hspace{3mm}\textit{State entities}            & 43     & 22    & 38        & 48      & 36        & 47    \\
\hspace{3mm}\textit{n}                         & 199    & 9     & 16        & 44      & 69        & 90    \\ \bottomrule
\end{tabular}
\end{table}

\end{document}
